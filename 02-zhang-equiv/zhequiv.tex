\documentclass[libertine,widepage]{lmaths}
\addbibresource{../ors.bib}
\ExecuteBibliographyOptions{url=false}

\title{Zhang's two characterisations of the notion of `minimal words' in a special monoid are equivalent}

\begin{document}

\maketitle

Let $M = \langle A \mid w = \epsilon \rangle$ be a one-relator monoid, with associated Thue congruence $T = \{(w, \epsilon)\}$. In his two papers \cite{Zhang1992}, \cite{Zhang1992a}, Zhang proves in two slightly different ways that the word problem for such a monoid is decidable, a result first shown by Adjan \cite{Adian1966}. Both involve constructing a monoid presentation for the group of units of $M$, to which the word problem for the whole monoid can then be reduced thanks to Magnus' theorem that the word problems of one-relator groups is decidable \cite{Magnus2004}. In both presentations, which are equivalent, the generating set comprises so-called \emph{minimal words}, arising from factors of the relator $w$.

The first paper \cite{Zhang1992a} gives a constructive definition: given a subset $X \subseteq A^*$ of words, define
	\begin{align*}
		L(X) &= \{ x \in A^* \mid \exists\ w \in A^* \colon wx \in X \} \text{ and } \\
		R(X) &= \{ x \in A^* \mid \exists\ w \in A^* \colon xw \in X \}
	\end{align*}
to be the set of left and right factors respectively of words in $X$. Then, if $W(X) = L(X) \cap R(X)$ is the set of words which are both left and right factors, let $W'(X)$ be those two-sided factors for which no proper right factor appears in $W(X)$.

Taking $C_1 = \{w\}$, put
	\[ C_{i+1} = C_i \cup \{ xy \mid y \in W(X), yx \in C_i \} \cup \{ yz \mid y \in W(X), zy \in C_i \} \]
Each word in $C_i$ has the same length as $w$: this is easy to see by induction on $i$. It follows from this that since $|C_i| \le |A|^{|w|}$ (which is finite since $M$ is finitely presented) and $C_{i+1} \supseteq C_i$ for all $i \ge 1$, there is a maximal set $C_k$. Let $E(M)$ equal $W'(C_k)$. Then the generating set for the group of units is an alphabet in bijection with $E(M)$.

In the second paper \cite{Zhang1992}\footnote{Note that the notation here has been simplified for the one-relation case, whereas the cited paper treats presentations with arbitrarily many relations of the form $w_i = \epsilon$.}, the author defines a minimal word as a word $u \in A^*$ of length at most $|w|$, none of whose proper left factors are invertible modulo the relation $\{(w, \epsilon)\}$. The set of all minimal words is named $D$. Since the word $w$ is invertible, it is asserted that it can be decomposed uniquely as a product $w = d_1 d_2 \cdots d_n$ of minimal words. For each factor, let $\Delta_s = \{ x \in D \mid x \leftrightarrow_T^* d_s \}$ be the set of words equivalent to $d_s$ in $M$. Then define $W = \{\Delta_1, \ldots, \Delta_n\}$ and $\Delta = \bigcup_{s = 1}^n \Delta_s$. The group of units is generated by an alphabet $B$ in bijection with $W$ under a homomorphism $\Delta^* \to B^*$ induced by sending letters in $\Delta_i$ to the letter in $B$ corresponding to $\Delta_i$.

Note that this alphabet $B$ is also in bijection with the minimal factors $d_i$, and we can define an equivalent homomorphism, so we shall consider the set $\Gamma = \{d_1, \ldots, d_n\}$ instead. Our aim is to show that $\Gamma = E(M)$, which we shall do through the following series of propositions.

\begin{lemma}
	Suppose a word $w \in C_i$ for some index $i$. Then $w$ is invertible modulo T.
\end{lemma}
\begin{proof}
	We proceed, as we will several times, by induction on $i$. The only word in $W_1$ is $w$, which is certainly invertible. Suppose all words in $C_1, C_2, \ldots, C_i$ are invertible, and consider $u \in C_{i+1}$. Suppose $u \not\in C_i$; then either $u = xy$ for some $y \in W(C_i)$ with $yx \in C_i$ or $u = yz$ for some $y \in W(C_i)$ and $zy \in C_i$. In the former case, by the inductive hypothesis, $yx$ is invertible and so $y$ is left invertible.
\end{proof}


\end{document}
