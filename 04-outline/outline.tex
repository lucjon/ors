\documentclass[showlabels,noindex,12pt]{lmaths}
\addbibresource{../ors.bib}
\ExecuteBibliographyOptions{url=false}

\usepackage{tikz}
\usetikzlibrary{arrows}
\usetikzlibrary{cd}
\tikzcdset{arrow style=math font}

\setlist{leftmargin=3em}

\let\oldbeginabstract\abstract
\renewcommand{\abstract}{\oldbeginabstract\noindent}

\newcommand{\draftnote}[1]{\textcolor{red}{#1}}
\newcommand{\nearrowstar}{\mathclap{\nearrow}_{*}}
\newcommand{\searrowstar}{\mathclap{\searrow}^{*}}

\DeclareMathOperator{\WP}{WP}
\DeclareMathOperator{\FG}{FG}

\title{The Word Problem for One-Relation Semigroups}
\author{Lucas Jones}

\begin{document}

\maketitle
\bigskip
\tableofcontents

\section{Introduction} \label{sec:intro}

The aim of this project is to investigate the decidability of the word problem for so-called special one-relation monoids, via Zhang's application of string rewriting systems. In \cref{sec:intro}, we will recall some elementary definitions which will be useful throughout the project, before giving a brief history of the word problem for semigroups and, in particular, an overview of work which has been done for the special case of one-relator semigroups. In \cref{sec:rewriting-systems}, we provide an elementary introduction to string rewriting systems and their properties, which is crucial to \cref{sec:special-monoids}, an exposition of Zhang's proof in \cite{Zhang1992a} that the word problem for one-relation special monoids is decidable.

It is expected that the reader has a basic familiarity with the theory of semigroups, equivalent to the St Andrews module \emph{MT5826 Semigroups} or equivalent.

\subsection{Free semigroups, monoids and groups}

An \dn{alphabet} is any finite set. The \dn{free semigroup} over this alphabet, denoted $A^{+}$, is the set of all nonempty finite strings made up of letters from $A$, equipped with the operation of concatenation.

The \dn{free monoid} $A^*$ over $A$ is the set $A^+ \cup \{\epsilon\}$ under juxtaposition, where $\epsilon$ is the empty string. A \dn{language} (over $A$) is any subset of $A^*$.

The construction of a free group is more involved, since the existence of inverses forces there to be some relations between letters. Again, let $A$ be an alphabet and fix an arbitrary set $A^{-1}$ which is disjoint but in bijection with $A$. We will write $a^{-1}$ for the image of $a \in A$ under this bijection. For any letter $a \in A$, the strings $aa^{-1}$ and $a^{-1}a$ are known as \dn{trivial relators}.

We say that two words $u$ and $v$ in $(A \cup A^{-1})^*$ are related by an elementary move if one can transform $u$ into $v$ by adding or removing a trivial relator at some point in $u$. We extend this to an equivalence relation $\sim$ by saying that $u \sim v$ if $u$ is related to $v$ by a finite sequence of elementary moves.  Furthermore, we claim that this relation $\sim$ is a congruence, and that the resultant quotient monoid $(A \cup A^{-1})^*/{\sim}$ is a group, called the \dn{free group} on $A$ and denoted $\FG(A)$.

\subsection{Presentations}

Presentations are a concise way of defining a semigroup, monoid or group in terms of a set of generators and a set of relations between these generating elements.

\begin{defn}
	A \dn{presentation} is an alphabet $A$ together with a set of `relations' $R \subseteq A^+ \times A^+$. Normally this is written $\langle A \mid R\rangle$, or $\langle A \mid a_1 = b_1, a_2 = b_2, \cdots \rangle$, if $R = \{(a_1, b_1), (a_2, b_2), \ldots\}$.
\end{defn}

\begin{defn}
	If $\langle A \mid R \rangle$ is a presentation, the \dn{semigroup presented by} $\langle A \mid R\rangle$ is quotient semigroup $A^+/\rho$, where $\rho$ is the smallest congruence on $A^+$ containing $R$. This is also denoted $\langle A \mid R \rangle$.
\end{defn}

For example, the free semigroup on two letters $F_2$ is presented by $\langle a, b \mid \rangle$. Similarly, the presentation $\langle a \mid a^7 = a \rangle$ defines the cyclic group of order 6.

This definition extends very naturally to monoids.

\begin{defn}
	The \dn{monoid presented by a presentation} $\langle A \mid R\rangle$ is the quotient monoid $\Mon \langle A \mid R\rangle = A^*/\rho$, where $\rho$ is the smallest congruence on $A^*$ containing $R$. This is also denoted $\langle A \mid R \rangle$.
\end{defn}

The bicyclic monoid, for example, can be written as $\Mon \langle b, c \mid bc = \epsilon \rangle$. This is equivalent to the semigroup presentation $\langle b, c, \epsilon \mid bc = \epsilon, b\epsilon = \epsilon b = b, a\epsilon = \epsilon a = a, \epsilon^2 = \epsilon \rangle$.

A group presentation is defined in a similar way to the above, except that we need to include the trivial relations $aa^{-1} = a^{-1}a = \epsilon$ implied by the group axioms.

\begin{defn}
	The \dn{group presented by a presentation} $\langle A \mid R\rangle$, is the quotient monoid $\Gp \langle A \mid R\rangle = A^*/\rho$, where $\rho$ is the smallest congruence on $A^*$ containing $R$ and the trivial relations, namely $aa^{-1} = e$ and $a^{-1}a = \epsilon$ for each $a \in A$.
\end{defn}

To provide a final example, the dihedral group of order 6 is given by the presentation $\Gp \langle \rho, \sigma \mid \rho^3 = \epsilon, \sigma^2 = \epsilon, \sigma\rho = \rho\sigma^{-1} \rangle$.

In this project, we will be interested mostly in \dn{finitely presented} semigroups, monoids and groups; that is, those structures which admit a presentation $\langle A \mid R\rangle$ where both $A$ and $R$ are finite. Not every semigroup is finitely presented, however. In particular, $\langle a, b \mid ab^ia = aba \quad\forall i \in \mathbb{N} \rangle$ defines a semigroup which is not finitely presented.

\subsection{Decidable and undecidable problems}

%In the 1930s, the discovery that several natural models of computation (such as Turing machines and Church's $\lambda$-calculus among others) were equally powerful provided strong motivation for most mathematicians to accept that for a procedure to be able to be solved by an algorithm meant that it could be solved by, for example, a Turing machine. With a concrete definition of computability came problems which were proven not to be computable. Turing described his machines for the first time in the 1936 paper \emph{On computable numbers with an application to the \emph{Entscheidungsproblem}}, which he used to prove the existence of these undecidable problems. He showed that the so-called halting problem was undecidable.

Fix an alphabet $A$ and let $L$ be a language over $A$. Then the \dn{decision problem} for $L$ takes as input a word $w$ in $A^*$ and outputs `true' if $w$ belongs in $L$ and `false' if it does not. A decision problem is said to be \dn{decidable} if there is an algorithm which always answers true or false in a finite number of steps, no matter what word $w$ is given as input.

One famous decision problem, which motivated much study in the early 20th century, is the \emph{Entscheidungsproblem} proposed by Hilbert in the 1920s. Roughly speaking, the language $L$ in this case is the set of logical statements, written in terms of an appropriate language, which are true assuming some set of axioms. Hilbert's challenge was to determine whether or not this problem is decidable.

Church and Turing proved (independently) in the 1930s that it is not. However, to do so, they had to formalise the notion of `algorithm'. The intuitive concept of an algorithm has, of course, existed for thousands of years --- however, in order to prove that none exists for a particular problem, a mathematically rigorous definition is required. Until this was available, questions such as Hilbert's, as well as those about problems associated with groups and semigroups which we will discuss in \hyperref[sec:word-problem]{the next section}, could not be answered negatively. The word problem for a group was defined long before this, but the only results concerning it were algorithms to solve it for particular classes of groups. It was only afterwards that it could be proven that the word problem is undecidable in general, for example. 

A more elementary example of an undecidable problem is the \dn{halting problem}. As input, it takes a description of an algorithm and an input word $w$. In return, it outputs `true' if the algorithm completes in a finite number of steps --- or \dn{halts} --- given the input $w$, and `false' if it runs indefinitely.

\begin{theorem}
	The halting problem is undecidable.
\end{theorem}
\begin{proof}
	Suppose the halting problem were decidable, so that there exists an algorithm $H(A, X)$ which returns in finite time whether or not the algorithm $A$ halts given input $X$. Define a new algorithm $S(A)$, as follows: 
	\begin{enumerate}	
		\item Compute $H(A, A)$, i.e. whether the algorithm $A$ halts given a description of $A$ as input.
		\item If $A$ does halt on input $A$, loop forever.
		\item Otherwise, output `true' and halt.
	\end{enumerate}

	Now consider the result of applying $S$ to a description of itself. If $S$ halts on $S$, then $H(S, S)$ must have been false. But this means that $S$ does not halt on input $S$, which is a contradiction. By assumption, our algorithm $H$ never halts. So if $S$ does not halt on $S$, $H(S, S)$ must be false --- but this is also a contradiction! So such an algorithm $H$ can not exist.
\end{proof}


\subsection{The word problem for groups} \label{sec:word-problem}

The word problem is one of three decision problems associated with a group introduced by Dehn in 1911, the others being the conjugacy problem and the isomorphism problem.

\begin{defn}
	Let $G = \Gp \langle A \mid R \rangle$ be a group. Then the \dn{word problem} for $G$, denoted $\WP(G)$, is the set
	\[ \WP(G) = \{ (u, v) \mid u, v \in A^*, u =_G v \} \]
	of pairs of words in the generators $A$ which are equal in $G$. The \dn{conjugacy problem} for $G$ is the set
	\[ \{ (u, v) \mid u, v \in A^*, \exists\ w \in G \colon u = w^{-1}vw \} \]
	of pairs of words which represent conjugate elements.
\end{defn}

The isomorphism problem is slightly different: it asks, given two presentations $\langle A \mid R \rangle$ and $\langle B \mid S \rangle$, whether the groups they present are isomorphic.

Each of these problems is readily generalisable to finitely presented semigroups. Axel Thue discussed, though in different language, the word problem for semigroups in his 1914 paper, \emph{Probleme über Veränderungen von Zeichenreihen nach gegebenen Reglen}.

At the time these questions were first posed, there was as yet no rigorous definition of `algorithm', and so no way of proving that the word problem for a particular group is unsolvable. The first results on the word problem were therefore positive ones, giving algorithms for particular special cases --- for example, in his 1911 paper \emph{Über unendliche diskontinuerliche Gruppen} which first introduced the problems, Dehn solved the word problem for some groups arising in topology.

The word problem is readily solvable for certain classes of groups. For example, if $G$ is a free group, the algorithm is as follows: suppose $G$ is freely generated by a set $A$, and suppose that we have two elements $u$ and $v$ of $G$, written as words in $A$ and the corresponding inverse letters. Then since there are no relations in the group except those that follow from the group axioms, namely $aa^{-1} = a^{-1}a = \epsilon$ for each $a \in A$, if we remove all of these pairs then $u =_G v$ if and only if these `reduced' words are equal, letter for letter.

In 1947, however, Emil Post proved that the general word problem for finitely-presented semigroups is undecidable; and shortly afterwards, in 1950, Turing himself showed that the word problem for cancellative finitely-presented semigroups is undecidable. The 1950s saw Novikov and Boone independently prove that the word problem for a general finitely-generated group is undeciadble.

Some important classes of groups and semigroups do, however, have decidable word problems. All finite groups and semigroups have decidable word problem (in fact, Anisimov showed that their word problem is regular). The special case which we shall be most concerned with in this project is that of the word problem of one-relation groups and semigroups.

\subsection{One-relation groups and semigroups}

A one-relation group or semigroup is a finitely presented group or semigroup which admits a presentation of the form $\langle A \mid u = v\rangle$, where $u$ and $v$ are words in $A^*$. The word problem for one-relator groups was in fact solved by Magnus in the early 1930s.

\begin{theorem}[Magnus, 1932] \label{thm:orgp-decidablewp}
	Let $G$ be a group admitting a presentation of the form $\langle A \mid u = v\rangle$, where $u$ and $v$ are words over the generating set $A$. Then the word problem for $G$ is decidable.
\end{theorem}

In his proof, Magnus relies heavily on the following result on the structure of one-relator groups, known as the \emph{Freiheitssatz}:

\begin{theorem}[Freiheitssatz] \label{thm:freiheitssatz}
	Let $G = \Gp \langle A \mid w = \epsilon \rangle$ be a one-relator group, with $w$ cyclically reduced, and let $a \in A$ be a letter contained in $w$. Then the subgroup $\langle A \setminus \{a\} \rangle$ is a free group.
\end{theorem}

Here, a word is said to be \dn{cyclically reduced} if all cyclic permutations of it are freely reduced. Modern proofs of the above theorems can be found in \cite{Lyndon2001}.

While the word problem for one-relator groups is in a sense completely solved, it is currently an open problem as to whether the word problem for one-relator semigroups is decidable. In this project, we consider a significant special case first proved by Adjan in the 1960s:

\begin{defn}
We say a one-relator monoid is \dn{special} if it admits a finite presentation of the form $\langle A \mid r = \epsilon \rangle$.
\end{defn}

\begin{theorem}[Adjan]
	Let $M$ be a one-relator special monoid. Then $M$ has decidable word problem.
\end{theorem}

There is a monoid analogue to the Freiheitssatz, due to Squier and Wrathall (\cite{Squier1983}):
% XXX check this XXX
\begin{theorem}[Freiheitssatz for one-relator monoids]
	Let $S = \langle A \mid u = v \rangle$ be a one-relator group, with $w$ cylically reduced, and let $a \in A$ be a letter contained in $uv$. Then the submonoid $\langle A \setminus \{a\} \rangle$ is free.
\end{theorem}
However, perhaps surprisingly, while both Zhang's proof and Adjan's original proof of the result for one-relator special monoids relies crucially upon the Freiheitssatz for groups, it does not use the monoid result at all.


\section{Rewriting systems} \label{sec:rewriting-systems}

\subsection{Introduction}

\begin{itemize}
	\item[Definition] string rewriting system; rewriting relation
\item[Definition] reflexive; transitive closure
	\item[Definition] descendant; common descendant
	\item[Definition] irreducible
	\item[Definition] equivalent; equivalent modulo a congruence
	\item[Definition] the rewriting algorithm
\end{itemize}

\subsection{Noetherian rewriting systems}

A noetherian rewriting system is one whose rewriting algorithm always terminates, no matter which string it is applied to. It is very useful to know that a rewriting system is noetherian if one wants to prove, as we will later, that an algorithm which uses it also always terminates. Before discussing noetherianness, we need to make some definitions relating to partial orders.

\begin{defn}
	A linear order on a set $X$ is an order $<$ on $X$ which has the following properties for all $a, b, c \in X$:
	\begin{itemize}
		\item \emph{Trichotomy:} exactly one of $a < b$, $b < a$ or $a = b$ holds.
		\item \emph{Transitivity:} if $a < b$ and $b < c$, then $a < c$.
	\end{itemize}
\end{defn}

\begin{defn}
	An order $<$ on a set $X$ is \dn{well-founded} if there is no sequence $x_1, x_2, \ldots \in X$ such that
	\[ x_1 < x_2 < x_3 < \cdots. \]
\end{defn}

An interesting application of well-foundedness is the concept of noetherian induction, which generalises regular mathematical induction to any set equipped with a well-founded order.

\begin{prop}[Noetherian induction] \label{prop:noetherian-induction}
	Let $P(x)$ be a property of elements of a set $X$, and suppose that $\rightarrow$ is a well-founded order on $X$. Then $P(x)$ is true for all $x \in X$ if for any $y \in X$, $P(y)$ is true if $P(z)$ is true for all $z \ne y$ such that $y \rightarrow^* z$. That is, if $P(y)$ is true whenever $P$ is true for every descendant of $y$.
\end{prop}
\begin{proof}
	Suppose that for each $y \in X$, $P(y)$ is true if $P(z)$ is true for all descendants $z$ of $y$, and suppose there is some $x \in X$ such that $P(x)$ is false. If $x$ has no descendants, then $P(x)$ is vacuously true. So $x$ must have a descendant $x_2$ such that $P(x_2)$ is false. By the same reasoning $x_2$ must have a descendant $x_3$ such that $P(x_3)$ is false. Continuing in this fashion, we have an infinite chain of descendants
		\[ \cdots \to x_3 \to x_2 \to x, \]
	for each of which $P$ is false. But this contradicts $\to$ being well-founded, since by definition a well-founded relation has no infinite chains.
\end{proof}

The usual notion of induction on the natural numbers is the same as the above if $\mathbb{N}$ is equipped with its natural order, which is clearly well-founded.

We can then define a noetherian rewriting system in terms of its reduction relation.

\begin{defn}
	A noetherian rewriting system is a rewriting system whose reduction relation $\rightarrow$ is well-founded.
\end{defn}


\begin{itemize}
	\item[Proposition] In a noetherian rewriting system, every string has an irreducible descendant.
	\item[Definition] (short-)lexicographic order
	\item[Proposition] The short-lex order is well-founded.
\end{itemize}

\subsection{Confluence}

\begin{defn}
	A rewriting system $R$ over an alphabet $A$ is \dn{confluent} if for all strings $a, b, c \in A^*$, whenever $a \to_R^* b$ and $a \to_R^* c$, there exists a string $z \in A^*$ such that $b \to^*_R z$ and $c \to^*_R z$.
\end{defn}

\begin{defn}
	A rewriting system $R$ over an alphabet $A$ is \dn{locally confluent} if for all strings $a, b, c \in A^*$, whenever $a \to_R b$ and $a \to_R c$, there exists a string $z \in A^*$ such that $b \to^*_R z$ and $c \to^*_R z$.
\end{defn}

\begin{theorem}[Newman] \label{thm:newman}
	If $R$ is a noetherian rewriting system, then $R$ is confluent if and only if it is locally confluent.
\end{theorem}

This theorem is often known as Newman's lemma, having first been proven by Newman in \cite{Newman1942}. Here we follow a simpler proof due to Huet written using the modern language of rewriting systems, which appears in \cite{Huet1980}. The principal technique of this proof is the notion of Noetherian induction (\cref{prop:noetherian-induction}).

\begin{proof}[ \pCref{thm:newman}]
	It is clear that a confluent system is locally confluent, since if $x \to z$ then $x \to^* z$ for all $x, z \in A^*$.

	Conversely, suppose $R$ is a locally confluent rewriting system over an alphabet $A$. We shall prove the statement $P(x)$ over $A^*$ by noetherian induction, defining $P(x)$ to be true if and only if for all $X, X' \in A^*$ such that $x \to^* X$ and $x \to^* X'$, there exists $z \in A^*$ such that $X \to^* z$ and $X' \to^* z$. If $P(x)$ holds for all strings $x$, the system is confluent by definition. 

	Let $x \in A^*$ be such that $P(x)$ holds for all its descendants. Let $X, X' \in A^*$ be such that $x \to_R^* X$ and $x \to_R^* X'$. In particular, because $\to$ is noetherian, there exist finite derivations $x = x_1 \to_R x_2 \to_R \cdots \to_R x_m = X$ and $x = x_1' \to_R x_2' \to_R \cdots \to_R x_n' = X'$. Since $R$ is locally confluent, the fact that $x \to_R x_1$ and $x \to_R x_1'$ means there exists a string $u \in A^*$ such that $x_1 \to_R^* u$ and $x_1' \to_R^* u$:

	{\centering
	\begin{tikzcd}
		& \arrow[dl] x \arrow[dr] \\
		x_1 \arrow[dd, "*"'] \arrow[dr, "*"] & & x_1' \arrow[dd, "*"] \arrow[dl, "*"'] \\
		& u & \\
		X & & X'
	\end{tikzcd}
	\par}

	Since $x_1$ is a descendant of $x$, by the inductive hypothesis, there exists a $v \in A^*$ such that $u \to^* v$ and $X \to^* v$.

	{\centering
	\begin{tikzcd}
		& \arrow[dl] x \arrow[dr] \\
		x_1 \arrow[dd, "*"'] \arrow[dr, "*"] & & x_1' \arrow[dd, "*"] \arrow[dl, "*"'] \\
		& u \arrow[ddl, "*"] & \\
		X \arrow[d, "*"'] & & X' \\
		v
	\end{tikzcd}
	\par}

	Next, since $x_1'$ is a descendant of $x$, by the inductive hypothesis there is a string $w \in A^*$ such that $u \to^* w$ and $X' \to^* w$.

	{\centering
	\begin{tikzcd}
		& \arrow[dl] x \arrow[dr] \\
		x_1 \arrow[dd, "*"'] \arrow[dr, "*"] & & x_1' \arrow[dd, "*"] \arrow[dl, "*"'] \\
		& u \arrow[ddl, "*"] \arrow[ddr, "*"'] & \\
		X \arrow[d, "*"'] & & X' \arrow[d, "*"] \\
		v & & w
	\end{tikzcd}
	\par}

	To complete the proof, we observe that $u$ is a descendant of $x$, so there is a string $z \in A^*$ such that $v \to^* z$ and $w \to^* z$.

	{\centering
	\begin{tikzcd}
		& \arrow[dl] x \arrow[dr] \\
		x_1 \arrow[dd, "*"'] \arrow[dr, "*"] & & x_1' \arrow[dd, "*"] \arrow[dl, "*"'] \\
		& u \arrow[ddl, "*"] \arrow[ddr, "*"'] & \\
		X \arrow[d, "*"'] & & X' \arrow[d, "*"] \\
		v \arrow[dr, "*"] & & w \arrow[dl, "*"'] \\
		& z & 
	\end{tikzcd}
	\par}

	Thus we have found a $z$ such that $x_1 \to^* z$ and $x_1' \to^* z$ as required.
\end{proof}

In order to show that a rewriting system is confluent, we can use the following condition, a restatement of theorem 1 in \cite{McNaughton1987}:
\begin{theorem} \label{thm:confluent-cond}
	A rewriting system $R$ over an alphabet $A$ is locally confluent if and only if the following hold for all strings $u, v, w, x, y \in A^*$:
	\begin{enumerate}[(1)]
		\item \label{it:conf-overlap} If $uv \to x$ and $vw \to y$ are rules of $R$ then there is a $z \in A^*$ such that $xw \to^*_R z$ and $uy \to^*_R z$.
		\item \label{it:conf-middle} If $uvw \to x$ and $v \to y$ are rules of $R$ then there is a $z \in A^*$ such that $x \to^*_R z$ and $uyw \to^*_R z$.
	\end{enumerate}
\end{theorem}

In fact, a rewriting system is confluent, not just locally confluent, if it satisfies the above condition. However, local confluence is sufficient for our purposes because the rewriting system we will consider in \cref{sec:special-monoids} is noetherian so we can apply \cref{thm:newman}.

\begin{proof}[ \pCref{thm:confluent-cond}]
	The `only if' direction is straightforward. To show the `if' direction, suppose $a, x, y$ are strings such that $a \to_R x$ through a rule $u \to v$ and $a \to_R y$ through a rule $u' \to v'$. Then $a$ contains $u$ as a substring, i.e. $a = AuZ$ for some $A, Z \in A^*$. The string $u'$ must also occur somewhere in $a$. There are five possibilities for its location, and in each case we show that a common descendant $z \in A^*$ exists for $x$ and $y$ and hence that $R$ is locally confluent.

	\textbf{Case (i)}: $u'$ is entirely contained in $A$. Write $a = Bu'CuZ$, so that $A = Bu'C$. Then \[
		\begin{array}{lllll}
			& & Au'CvZ & & \\
			& \nearrow & & \searrow & \\
			a = Bu'CuZ & & & & Av'CvZ, \\
			& \searrow & & \nearrow & \\
			& & Av'CuZ & &
		\end{array}
	\]
	so $z = Av'CvZ$ works.

	\textbf{Case (ii)}: $u'$ is contained in $Au$ but not in $A$ or $u$. Factor $a = BCDEZ$ as follows, so that $u = DE$ and $u' = CD$:

	{\centering
	\begin{tikzpicture}
		\filldraw[fill=gray!20] (0,0) rectangle (2,0.5) node[midway] {$A$};
		\draw[<->] (0.05,0.6) -- (0.75,0.6) node[above,midway] {$B$};
		\draw[dashed] (0.8,0) -- (0.8,0.5);
		\draw[<->] (0.85,0.6) -- (1.95,0.6) node[above,midway] {$C$};
		\draw[solid] (2,0) -- (2,0.5);
		\filldraw[fill=gray!20] (2,0) rectangle (4,0.5) node[midway] {$u$};
		\draw[<->] (2.05,0.6) -- (2.75,0.6) node[above,midway] {$D$};
		\draw[<->] (0.85,-0.1) -- (2.75,-0.1) node [below,midway] {$u'$};
		\draw[dashed] (2.8,0) -- (2.8,0.5);
		\draw[<->] (2.85,0.6) -- (3.95,0.6) node[above,midway] {$E$};
		\filldraw[fill=gray!20] (4,0) rectangle (6,0.5) node[midway] {$Z$};
	\end{tikzpicture}\par}

	Then we have rules $CD \to v'$ and $DE \to v$ in $R$, so by hypothesis there is a $z \in A^*$ such that $v'E \to^*_R z$ and $Cv \to^*_R z$, and we have a common descendant like so: \[
		\begin{array}{lllll}
			& & Bv'EZ & & \\
			& \nearrow & & \searrowstar & \\
			a = BCDEZ & & & & BzZ. \\
			& \searrow & & \nearrowstar & \\
			& & BCvZ & &
		\end{array}
	\]

	\textbf{Case (iii)}: $u'$ is entirely contained in $u$. Factor $a = ABu'CZ$ so that $u = Bu'C$. Then by hypothesis there is a $z \in A^*$ such that $v \to^* z$ and $Bv'C \to^* z$. So we have the diagram: \[
		\begin{array}{lllll}
			& & ABv'CZ & & \\
			& \nearrow & & \searrowstar & \\
			a = ABu'CZ & & & & AzZ. \\
			& \searrow & & \nearrowstar & \\
			& & AvZ & &
		\end{array}
	\]

	\textbf{Case (iv)}: $u'$ is contained in $uZ$ but not in $u$ or $Z$. Factor $a = ABCDE$ as below, so that $u = BC$, $u' = CD$ and $Z = DE$:

	{\centering
	\begin{tikzpicture}
		\filldraw[fill=gray!20] (0,0) rectangle (2,0.5) node[midway] {$A$};
		\draw[<->] (2.05,0.6) -- (2.75,0.6) node[above,midway] {$B$};
		\draw[<->] (2.85,0.6) -- (3.95,0.6) node[above,midway] {$C$};
		\draw[solid] (2,0) -- (2,0.5);
		\filldraw[fill=gray!20] (2,0) rectangle (4,0.5) node[midway] {$u$};
		\draw[<->] (4.05,0.6) -- (4.75,0.6) node[above,midway] {$D$};
		\draw[dashed] (2.8,0) -- (2.8,0.5);
		\draw[<->] (4.85,0.6) -- (5.95,0.6) node[above,midway] {$E$};
		\filldraw[fill=gray!20] (4,0) rectangle (6,0.5) node[midway] {$Z$};
		\draw[dashed] (4.8,0) -- (4.8,0.5);
		\draw[<->] (2.85,-0.1) -- (4.75,-0.1) node [below,midway] {$u'$};
	\end{tikzpicture}\par}

	Since we have rules $BC \to v$, $CD \to v'$, by hypothesis there is a $z \in A^*$ such that $vD \to^* z$ and $Bv' \to z$, giving the diagram: \[
		\begin{array}{lllll}
			& & AvDE & & \\
			& \nearrow & & \searrowstar & \\
			a = ABCDE & & & & AzZ. \\
			& \searrow & & \nearrowstar & \\
			& & ABv'E & &
		\end{array}
	\]

	\textbf{Case (v)}: $u'$ is entirely contained in $Z$. This case is very similar to case (i) and is omitted.
\end{proof}


\begin{itemize}
	\item[Proposition] In a confluent rewriting system a string has at most one irreducible descendant.
\end{itemize}

\subsection{Normal forms}

\begin{itemize}
	\item[Proposition] Confluent noetherian rewriting systems give unique normal forms.
	\item When is the normal form computable?
\end{itemize}

\section{The word problem for one-relation special monoids} \label{sec:special-monoids}

In this section, we discuss the word problem for a special class of one-relation semigroups, namely `special' one-relation monoids.

\begin{defn} \label{def:special}
	A monoid $M$ is \dn{special} if it has a presentation of the form $\langle A \mid w_1 = \epsilon, w_2 = \epsilon, \ldots, w_n = \epsilon \rangle$ for nonempty words $w_i \in A^*$.
\end{defn}

The main result is a theorem of Adjan, which states that the word problem is indeed decidable for one-relation special monoids:

\begin{theorem}[Adjan] \label{thm:ors-decidablewp}
	Let $M$ be a one-relation special monoid, i.e. a monoid admitting a presentation of the form $\langle A \mid w = \epsilon\rangle$, where $A$ is an alphabet, $w \in A^+$. Then $M$ has decidable word problem.
\end{theorem}

Here, we follow Zhang's proof from \cite{Zhang1992a}. Zhang's overall approach is to construct a one-relation presentation for the group of units of the special monoid --- whose word problem is then decidable by Magnus' theorem --- and reduce the word problem of the whole monoid to that of the group of units using a confluent, noetherian rewriting system.

In another paper (\cite{Zhang1992}), Zhang uses a similar method to generalise this statement by reducing the word problem of any special monoid to the word problem of its group of units, as well as obtaining similar results on the conjugacy and divisibility problems for these monoids.


\subsection{Constructing a generating set}

Suppose $M = \langle A \mid w = \epsilon\rangle$ is a finitely presented monoid.

Given a subset $X \subseteq A^*$ of words, define
	\begin{align*}
		L(X) &= \{ x \in A^+ \mid \exists\ w \in A^* \colon wx \in X \} \text{ and } \\
		R(X) &= \{ x \in A^+ \mid \exists\ w \in A^* \colon xw \in X \}
	\end{align*}
to be the set of left and right factors respectively of words in $X$, and let $W(X) = L(X) \cap R(X)$ be the set of words which are both left and right factors each of some word in $X$.

Taking $C_1 = \{w\}$, let
	\[ C_{i+1} = C_i \cup \{ xy \mid y \in W(C_i), yx \in C_i \} \cup \{ yz \mid y \in W(C_i), zy \in C_i \} \]
be those words obtained by moving right factors from $W(C_i)$ to the beginning of the words in $C_i$ in which they appear and left factors to the end of words.

\begin{lemma}
	For each index $i$, every word in $C_i$ has the same length as $w$.
\end{lemma}
\begin{proof}
	This can be shown by a straightforward induction on $i$. It is true in $C_1$, since $C_1 = \{w\}$. So suppose every word in $C_i$ has length $|w|$, and let $xy$ be a new word in $C_{i+1}$ for some $x, y \in A^+$ such that $x$ or $y$ is in $W(C_i)$. Then either $xy$ or $yx$ in $C_i$ and hence $|xy| = |yx| = |w|$ by the inductive hypothesis. Hence every word in $C_{i+1}$ has length $|w|$ and so every word in $C_i$ has length $|w|$, for any index $i$.
\end{proof}

It follows from the above that since $|C_i| \le |A|^{|w|}$ (which is finite since $M$ is finitely presented) and $C_{i+1} \supseteq C_i$ for all $i \ge 1$, there is a maximal set $C_k$. Define a new set $E(M)$ to be those words in $W(C_k)$ which have no proper right factors in $W(C_k)$.

This set $E(M)$ will effectively serve as the generating set in the presentation for the group of units of $M$ which we construct in the remainder of the proof.

\begin{example}
	Consider the monoid $M$ defined by the presentation $\langle b, c \mid bcb = \epsilon \rangle$. Running the procedure again we obtain:

	\begin{center}
	\renewcommand{\arraystretch}{1.2}
	\begin{tabular}{r|ll}
		$i$ & $C_i$ & $W(C_i)$ \\ \hline
		1 & $\{bcb\}$ & $\{b, bcb\}$ \\
		2 & $\{cbb, bbc, bcb\}$ & $\{cbb, c, b, bb, bc, cb, bbc, bcb\}$
	\end{tabular}
	\end{center}

	Notice that both generators $b$ and $c$ are in $W(C_2)$, so $C_2$ contains every cyclic permutation of $bcb$. Hence $C_k = C_2$ and so $E(M) = \{ b, c\}$. Since $E(M)$ is supposed to generate the group of units of $M$, this suggests (correctly) that every generator, and hence every element, of the monoid is invertible, i.e. that $M$ is in fact a group. 
\end{example}

\begin{example}
	Consider the bicyclic monoid, given by the presentation $B = \langle b, c \mid bc = \epsilon \rangle$.
	Applying the above construction, we find:

	\begin{center}
	\renewcommand{\arraystretch}{1.2}
	\begin{tabular}{r|ll}
		$i$ & $C_i$ & $W(C_i)$ \\ \hline
		1 & $\{bc\}$ & $\{bc\}$ \\
		2 & $\{bc\}$ & $\{bc\}$
	\end{tabular}
	\end{center}

	So $C_k = C_1 = \{ bc \}$, and hence $E(B) = bc$. This suggests that the only invertible element in $B$ is the identity $(bc)^n = bc = \epsilon$.
\end{example}

\begin{example}
	For a less extreme example, look at the monoid $N$ with defining presentation $\langle a, b, c \mid bcab = \epsilon\rangle$. Performing the computations:

	\begin{center}
	\renewcommand{\arraystretch}{1.2}
	\begin{tabular}{r|ll}
		$i$ & $C_i$ & $W(C_i)$ \\ \hline
		1 & $\{bcab\}$ & $\{b, bcab\}$ \\
		2 & $\{bbca, cabb, bcab\}$ & $\{b, bb, bbca, ca, bcab, bca, cabb, cab\}$
	\end{tabular}
	\end{center}

\end{example}

\subsubsection{An alternative definition}

In \cite{Zhang1992}, an alternative, non-constructive definition of a generating set for the group of units is given, based on the idea of \emph{minimal factors} of $w$.

\begin{defn}
	A word $v \in A^+$ is \dn{minimal} if:
	\begin{enumerate}[(i)]
		\item it is invertible in $M$,
		\item $|v| \le |w|$ and
		\item none of its nonempty prefixes are invertible in $M$.
	\end{enumerate}
\end{defn}

...


\subsection{Defining the presentation}

Define a new alphabet $B$ disjoint with $A$ such that we have a bijection $\midtilde\phi \colon E(M) \to B$, and let $\phi \colon E(M)^* \to B^*$ be the unique homomorphic extension of $\midtilde\phi$ to $E(M)^*$. By the following result, $\phi(w)$ is well-defined:

\begin{prop} \label{lma:relator-factors-E(M)}
	The relator $w$ is a product of factors in $E(M)$.
\end{prop}
\begin{proof}
	\hspace{-0.25mm}We show by induction on word length that every element of $W(C_k)$ is a product of factors in $E(M)$. First suppose that $u \in W(C_k)$ is of minimal length in $W(C_k)$. Then $u$ has no proper factors in $W(C_k)$, and so $u \in E(M)$.

	Now suppose that all strings in $W(C_k)$ with length less than $n$ are products of factors in $E(M)$, and let $u \in W(C_k)$ have length $n$. If $u$ has no proper right factors in $W(C_k)$, then it is in $E(M)$ by definition. Otherwise, we can write $u = ve$ for $e \in E(M)$, $v \in A^+$.

	We want to show that $v \in W(C_k)$, so that by the inductive hypothesis (since $|v| < n$), $v \in E(M)^*$ and hence $u \in E(M)^*$. Since $v$ is a prefix of $u$, for $v$ to be in $W(C_k)$ it remains only to show that it is a suffix of some word in $C_k$. Because $u \in W(C_k)$, there is some word $x \in C_k$ such that $x = x'u = x've$ . Then by the definition of $C_{k+1}$, since $e \in W(C_k)$, the word $ex'v \in C_{k+1}$. But $C_{k+1} = C_k$, so $v$ is indeed a suffix of a word in $C_k$.

	Altogether, this means $u$ is a product of factors in $E(M)$ and so by induction all the words of $W(C_k)$ are such a product. In particular, since $w \in W(C_k)$, $w$ is a product of factors in $E(M)$.
\end{proof}

We now define a monoid $M'$ given by the presentation $\langle B \mid \phi(w) = \epsilon \rangle$. We assert that this monoid is in fact a group:

\begin{prop}
	The monoid $M' = \langle B \mid \phi(w) = \epsilon\rangle$ is a group.
\end{prop}
\begin{proof}
	It suffices to show that every generator of $M'$ is invertible. We first show by induction that for all $i$, every word in $C_i$ is trivial in $M$. In the base case $C_1$, the only word to consider is $w$ itself, which is trivial by the defining relation $w = \epsilon$. Suppose then that all words in $C_j$ are trivial, for each $j \le i$, and consider a word $v \in C_{i+1}$, $v \not\in C_i$.

	This new word $v$ can arise in two ways. In the first case, $v = xy$ for some $y \in W(C_i)$, $yx \in C_i$. So by the inductive hypothesis, $yx =_M \epsilon$, i.e. $x$ is a right inverse for $y$ in $M$. As $y \in W(C_i)$, there is a word $u \in C_i$ such that $u = u'y =_M \epsilon$ for some $u' \in A^*$. So $u'$ is a left inverse for $y$ in $M$. We have a left and a right inverse for $y$ in $M$, so they must be equivalent in $M$, hence $v = xy =_M u'y =_M u$. But $u$ is in $C_i$ and so is trivial by the inductive hypothesis. Therefore $v =_M \epsilon$ as required.

	A similar argument shows that $v$ is trivial in $M$ if $v = yz$ for $y \in W(C_i)$, $zy \in C_i$. This accounts for every word in $C_{i+1}$, and so by induction all words in $C_j$ are trivial in $M$, for every index $j$. In particular, all the words in $C_k$ are trivial.

	For any word $z \in W(C_k)$, there are words $a, b$ in $C_k$ such that $a = za'$ and $b = b'z$ for some $a', b' \in A^*$. Since these words $a$ and $b$ are trivial in $M$, $a'$ is a right inverse for $z$ and $b'$ is a left inverse for $z$ in $M$ and hence $z$ is invertible in $M$. In particular, it follows that every word in $E(M)$ is invertible.

	To conclude, let $b \in B$ be a generator of $M'$. Then $\tilde\phi^{-1}(b) \in E(M)$ has an inverse $\midbar b$ in $M$. Hence $\epsilon =_M \tilde\phi^{-1}(b) \midbar{b}$ and so $\epsilon =_{M'} b\phi(\midbar{b})$; and likewise $\phi(\midbar{b})b = \epsilon$. The arbitrary generator $b$ of $M'$ is hence invertible, so $M'$ is a group.
\end{proof}

We assert further that it is, as suggested, the group of units of $M$.

\begin{theorem}
	The monoid $M'$ is isomorphic to the group of units of $M$.
\end{theorem}

In light of this, we shall henceforth refer to the monoid $M'$ as $G$.

\subsection{The rewriting system}

Let $\prec$ be some linear order on the alphabet $A$, and extend it to a shortlex order $<$ on $A^*$. Then define a rewriting system $R$ on $A^*$ by
	\[ R = \{ (u, v) \mid u, v \in E(M)^*, v < u, \phi(u) =_G \phi(v) \}. \]

We note immediately that $R$ is noetherian: since if $(u, v)$ is a rule in $R$, then we must have $v < u$, and the short-lex order $<$ is a well-founded relation by a result from the previous section.

This system $R$ is equivalent to the system $\{(w, \epsilon)\}$, i.e. they present the same monoid:
\begin{lemma} \label{lma:R-equivalent-to-pres}
	The rewriting systems $\{(w, \epsilon)\}$ and $R$ are equivalent in the sense that their induced congruences  are equal, and hence that $\langle A \mid w \rangle$ = $\langle A \mid \{u = v : (u, v) \in R\}\rangle$.
\end{lemma}
\begin{proof}
	First, observe that $w \in E(M)^*$ by \cref{lma:relator-factors-E(M)}; $\epsilon < w$; and $\phi(w) =_G \epsilon$ follows immediately from the presentation for $G$, so $(w, \epsilon)$ is a rule in $R$ and hence $\leftrightarrow^*_{\{(w,\epsilon)\}}\ \subseteq\ \leftrightarrow^*_R$.

Conversely, suppose $(u, v)$ is a rule in $R$. Then $\phi(u) =_G \phi(v)$ means $\phi(u) \leftrightarrow^*_{\{(\phi(w), \epsilon)\}} \phi(v)$, and since $\phi$ is a homomorphism, $u \leftrightarrow^*_{\{(w, \epsilon)\}} v$. So $\leftrightarrow^*_R\ \subseteq\ \leftrightarrow^*_{\{(w,\epsilon)\}}$.

Hence the two rewriting systems are equivalent.
\end{proof}


\subsubsection{\texorpdfstring{Some lemmas on $E(M)$}{Some lemmas on E(M)}}

We state here two lemmas about the generating set $E(M)$ which will be useful to prove that $R$ is confluent:

\begin{lemma} \label{lma:no-middle-E(M)}
	If $x, y, z \in A^*$ are strings such that $xy, yz \in E(M)$, then either $y = \epsilon$ or $x = z = \epsilon$.
\end{lemma}
\begin{proof}
	Suppose $y \ne \epsilon$. Then since $xy \in E(M)$, $xy \in W(C_k)$ and in particular, $xy \in L(C_k)$, so there is some $\alpha \in A^*$ such that $\alpha x \cdot y \in C_k$. So $y$ is a right factor of a word in $C_k$, i.e. $y \in R(C_k)$. Likewise, $yz \in R(C_k)$ so $y \cdot z\beta \in C_k$ for some $\beta \in A^*$ and $y \in L(C_k)$. Hence $y \in W(C_k)$. By the definition of $E(M)$, $xy$ has no proper right factor in $W(C_k)$, so $y$ must not be a proper factor; but by assumption, $y \ne \epsilon$, so we must have $x = \epsilon$. Similarly, $yz$ has no proper right factor in $W(C_k)$, so $z$ is not a proper factor and $z = \epsilon$.
\end{proof}

\begin{cly} \label{cly:middle-E(M)*}
	If $x, y, z \in A^*$ are strings such that $xy, yz \in E(M)^*$, then $y \in E(M)^*$.
\end{cly}


\subsubsection{\texorpdfstring{Confluence of $R$}{Confluence of R}}

We apply \cref{thm:confluent-cond} directly to $R$ to show that it is locally confluent. Since $R$ is noetherian, by \cref{thm:newman} this suffices to show $R$ is confluent.

For part \ref{it:conf-overlap}, suppose $uv \to_R x$ and $vw \to_R y$, with $uv \ne vw$ (if they are equal, we have nothing to prove). We claim that in fact one of $xw \to uy$ or $uy \to xw$ is a rule in $R$ and so $uy$ or $xw$ is a suitable $z$. By the definition of $R$, $uv, vw, x, y\in E(M)^*$, and so by \cref{cly:middle-E(M)*}, $v \in E(M)^*$. Hence $u, w \in E(M)^*$ and $uy, xw \in E(M)^*$. Now observe that by the rules we know are in $R$ and the fact $\phi$ is a homomorphism, we have $\phi(xw) = \phi(x)\phi(w) =_G \phi(uv)\phi(w) = \phi(u)\phi(vw) =_G \phi(u)\phi(y) = \phi(uy)$. Finally, since $<$ is linear, either $xw < uy$, in which case $uy \to xw$ is a rule in $R$; or $uy < xw$, in which case $xw \to uy$ is a rule in $R$ as required.

Now consider condition \ref{it:conf-middle}: suppose $uvw \to_R x$ and $v \to_R y$, and again assume that $v \ne y$. Observe that $\phi(x) =_G \phi(uyw)$, since $\phi(uvw) =_G \phi(x)$ and $\phi(v) =_G \phi(y)$. Furthermore, either $uyw < x$ or $x < uyw$. So as before, if $uyw \in E(M)^*$, then one of $x$ or $uyw$ will be a suitable $z$. Since $uvw, v, y \in E(M)^*$, if $uyw \not\in E(M)^*$, then $u = u_1u_2 \cdots u_m U$ and $w = W w_1 w_2 \cdots w_n$ for $u_i, w_i \in E(M)$ and some non-empty strings $U$ and $W$, with $UvW \in E(M)$. This means $v$ must be shorter than the longest word in $E(M)$ (or $UvW$ would be longer and therefore not in $E(M)$).

The following lemma completes the proof:

\begin{lemma} \label{lma:shorter-irreducible}
	Any word $a \in A^*$ shorter than the longest word in $E(M)$ is irreducible by $R$.
\end{lemma}
\begin{proof}
	Let $L$ be a word of maximal length in $E(M)$, and suppose $a$ were reducible. Then we can write $a = a'ba''$ where $b \in E(M)^*$ and $b \to_R c$ for some $c \in E(M)^*$. Then $c < b$ and so $|c| \le |b| \le |a| < |L|$, and hence $L$ cannot appear in $b$ or $c$.
	
	We then recall the Freiheitssatz for one-relator groups (\cref{thm:freiheitssatz}).	Since $b \to_R c$, $\phi(b) =_G \phi(c)$. Then as $L$ is not in $b$ or $c$, $\phi(L)$ is not in $\phi(b)$ or $\phi(c)$, i.e. $\phi(b)$ and $\phi(c)$ are in the subgroup of $G$ generated by $B \setminus \{\phi(L)\}$. This is free by the Freiheitssatz, and so $\phi(b) = \phi(c)$ as words. Since $\phi$ is a homomorphism and $\midtilde\phi$ is a bijection, $\phi$ is injective and so $b = c$. But $c < b$, so we have a contradiction and $a$ is indeed irreducible.
\end{proof}

It follows from this that $v$ is irreducible by $R$; but this is impossible, since by hypothesis $v \to_R y$. So in fact $uyw \in E(M)^*$, and one of $uyw \to_R x$ or $x \to_R uyw$ holds. Therefore condition \ref{it:conf-middle} is satisfied and $R$ is confluent.


\subsection{The word problem is decidable}

We can now prove \cref{thm:ors-decidablewp}. Suppose we have two words $u$ and $v \in A^*$. Observe that we can enumerate the rules of $R$ which could apply to $u$ in a finite number of steps: since if a rule $(a, b)$ applies to $u$, then $a$ is a substring of $u$ and so $|b| \le |a| \le |u|$. Enumerating all possible pairs $(a, b)$ of strings in of length $|u|$ or less, it is decidable whether $b < a$ and --- by \cref{thm:orgp-decidablewp} --- whether $a =_G b$, and hence whether $(a, b)$ is a rule in $R$. If there are any such rules, we can apply one and repeat the process with the resulting string. Eventually, since $R$ is confluent and noetherian, no such rules shall remain and we will have computed the unique normal form $\midbar{u}$ for $u$.

Repeating this process starting with $v$, we can compute its normal form $\midbar{v}$ in a finite number of steps. Since $R$ is equivalent to $\{(w, \epsilon)\}$ by \cref{lma:R-equivalent-to-pres}, $u =_M v$ if and only if $\midbar{u} = \midbar{v}$ as words. We can perform this comparison in a finite number of steps, so the overall problem of determining whether $u =_M v$ is decidable. This is precisely the word problem for $M$.


\printbibliography

\printindex

\end{document}
